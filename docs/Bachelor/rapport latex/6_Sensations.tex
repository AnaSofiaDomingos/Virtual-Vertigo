\chapter{Sensations}
Le réalisme est très important dans ce projet, c'est donc pour cela que utilisation de quelques appareils supplémentaire seront mis en place afin de plonger la personne le plus possible en situation. \\

%%%%%%%%%%%%%%%%%%%%%%%%%%%%%%%%%%%%%%%%%%%%%%%%%%%%%%%%%%%%%%%%%%%%%%%%%%%%%%%%%%%%%%%%%%%%%%%%%%%%%%%%%%%%%%%%%%%%%%%%%%%

\section{Planche}
Afin de pouvoir ressentir la planche sous nos pieds, une planche de 2,50m \- distance que le Kinect version 1 capte \- ayant environ 5 cm d'épaisseur a été mise en place. \\
Cette planche va permettre à la personne de sentir sous ses pieds les bords de la planche quand elle s'en approche et donc avoir la sensation de déséquilibre et de chute possible. 

%%%%%%%%%%%%%%%%%%%%%%%%%%%%%%%%%%%%%%%%%%%%%%%%%%%%%%%%%%%%%%%%%%%%%%%%%%%%%%%%%%%%%%%%%%%%%%%%%%%%%%%%%%%%%%%%%%%%%%%%%%%

\section{Vent}
Lorsque nous nous trouvons sur le toit d'un immeuble, il y a souvent du vent. C'est pour cette raison qu'un ventillateur sera ajouter afin la personne ressente le vent lors de l'exercice.

%%%%%%%%%%%%%%%%%%%%%%%%%%%%%%%%%%%%%%%%%%%%%%%%%%%%%%%%%%%%%%%%%%%%%%%%%%%%%%%%%%%%%%%%%%%%%%%%%%%%%%%%%%%%%%%%%%%%%%%%%%%

\section{Effet Sonore}
Lorsque nous nous trouvons sur le toit d'un immeuble, il y a également tout les bruits du quotidiens, comme par exemple, les klaxons, les voitures qui passent, les gens qui parlent ou même les oiseaux. \\
Un enregistrement des bruits sonores ambiant sera effectué et lorsque la personne se trouvera au bord de la planche, il se déclancherait. Pour cela, il y a 2 solution possible : \\

\begin{itemize}
\item Un casque bluetooth qui sera connecté à l'ordinateur
\item Le smartphone lui-même\\

\end{itemize}

Cette amélioration a été mise en place utilisant le casque bluetooth car sur le smartphone il est impossible de déclencher la bande sonore sans avoir un touch event. \\
Le problème sur smartphone n'a pas été vu pour une question de temps mais si le temps l'aurait permit, le lancement de la bande aurait été effecuté utilisant \color{red} à voir \color{black} \\

%%%%%%%%%%%%%%%%%%%%%%%%%%%%%%%%%%%%%%%%%%%%%%%%%%%%%%%%%%%%%%%%%%%%%%%%%%%%%%%%%%%%%%%%%%%%%%%%%%%%%%%%%%%%%%%%%%%%%%%%%%%

\section{Pulsations}
Afin de pouvoir comparer nos résultats en fonction de nos tentatives et donc savoir si notre système permet bien de vaincre la peur du vide, une montre connectée sera mise en place. L'utilisateur devra la porter lors de l'exercice. \\
Cette montre communiquera au serveur via le téléphone les informations cardiaques. Ces information serons stocker sur une base de données ou dans un fichier et elles permettrons de suivre l'évolution.\\
Elle permettera également de contrôler en temps réel la personne en cas de crise ou tout autre problème physique. 

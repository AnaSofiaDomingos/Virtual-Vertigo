\section*{Avant propos}

Ce rapport décrit mon travail de Bachelor, réalisé sur une période de 12 semaines au sein de hepia. Le sujet de ce travail de Bachelor, a été proposé par \textsc{M. Jérémy Gobet} assistant à hepia à la filière ITI est l'implémentation du projet \textit{Virtual-Vertigo}. Ce projet consiste à mettre en place un exercice pour aider les personnes à combattre la peur du vide en utilisant la réalité virtuelle avec des casques de réalité virtuelle en carton et un smartphone. A noter que ce travail est la continuité de mon projet de semestre qui consistait à la prise en main, à la recherche et au choix des technologies et outils du projet \textit{Virtual-Vertigo}. 

\subsection*{Impressions personnelles}
Ce travail a été très intéressant, amusant et attractif à mettre en place. En effet, ce travail de Bachelor touche des domaines que j'apprécie énormément (développement \textit{Web} et \textit{3D}). L'implémentation du projet \textit{Virtual-Vertigo} a été agréable et importante car à chaque test de simulation, tout le monde voulait tester également. Il intéresse énormément de gens par son innovation. En effet, il vend du rêve : l'immersion totale dans un monde virtuel. Mon but était d'avoir une simulation opérationnelle même si elle contient encore quelques problèmes. J'ai hâte de le montrer au plus de monde possible et de travailler dans les domaines abordés dans ce travail. Ce fut une expérience très enrichissante et instructive. \\

J'espère que ce rapport et ce projet sera aussi intéressant pour vous qu'il a été pour moi de le mettre en place et de l'écrire.
\begin{center} Je vous souhaite une bonne lecture. \end{center} 
%%%%%%%%%%%%%%%%%%%%%%%%%%%%%%%%%%%%%%%%%%%%%%%%%%%%%%%%%%%%%%%%%%%%%%%%%%%%%%%%%%%%%%%%%%%%%%%%%%%%%%%%%%%%%%%%%%%%%%%%%%%
\subsection*{Informations}


\textbf{António Domingos Ana Sofia :} \href{ana.domingos@hotmail.fr}{ana.domingos@hotmail.fr}

Etudiante ITI travaillant sur le projet \textit{Virtual-Vertigo}

\textbf{Albuquerque Paul :} \href{paul.albuquerque@hesge.ch}{paul.albuquerque@hesge.ch}

Professeur à hepia dans la filière ITI reponsable

\textbf{Gobet Jérémy : }\href{jeremy.gobet@hesge.ch}{jeremy.gobet@hesge.ch}

Assistant à hepia dans la filière ITI ayant proposé le sujet et suivant également ce travail 

\subsection*{Suivi} 
Afin de permettre aux personnes suivant le projet de suivre l'avancement du \textit{Virtual-Vertigo} à tout moment, un dépôt \textit{GitHub} a été mis en place. \textit{GitHub} est un service Web d'hébergement et de gestion de développement de logiciels, utilisant le logiciel de gestion Git. Il propose un système de \textsf{versionning} (cela peut éviter les catastrophes) et un flux donnant la possibilité de partager et suivre des projets. \\
A noter que le dépôt \textit{GitHub} a été créé lors du travail de semestre et un serveur d'hébergement a été mis en place. Ce serveur a été mis en place parce qu'il était impossible de tester la réalité virtuelle localement. Un système de \textit{WebHook} a été ajouté à \textit{GitHub} afin de pouvoir mettre à jour le serveur hébergeant lors des changements sur le dépôt GitHub. \textit{WebHook} est une méthode permettant d'étendre, de personnaliser et d'intégrer une application Web lorsqu'un événement se produit (pour plus d'informations sur les \textit{WebHook}, voir \cite{webhook}).\\
A noter également que le \textit{WebHook} n'est plus utilisé durant le travail de Bachelor. En effet, un serveur \textit{Web} a pris la relais (pour plus d'informations sur ce serveur, voir la section~\ref{serveur}).

\newpage 

\section*{Remerciements}

Je tiens à remercier toutes les personnes qui m'ont aidé à la réalisation de ce travail de Bachelor. Plus particulièrement : \\

\textsc{M. Paul Albuquerque}, \textsc{M. Jérémy Gobet} et \textsc{M. Michaël Polla}, professeur et assistants à hepia à la filière ITI, pour m'avoir conseillé et aidé tout au long de la réalisation du projet \textit{Virtual-Vertigo}. \\

\textsc{M. Olivier Donzé} et \textsc{M. Benjamin Dupond-Roy}, professeur et assistant à hepia à la filière Architecture du paysage, pour m'avoir aidé et modélisé une scène de réalité virtuelle pour le projet \textit{Virtual-Vertigo}. \\

\textsc{M. Pascal Regamey}, assistant technique à hepia au département construction et environnement, pour avoir fournis la planche de 2.5 mètres pour la simulation de \textit{Virtual-Vertigo}. \\

Mes collègues de diplômes, pour m'avoir aidé à résoudre des problèmes, pour effectuer des simulations, pour m'avoir consacré du temps et pour la bonne humeur au cours de ce travail de diplôme.